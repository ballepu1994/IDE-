\documentclass[10pt,a4paper]{report}
%\usepackage[latin1]{inputenc}
\usepackage[utf8]{inputenc}
\usepackage{amsmath}
\usepackage{amsfonts}
\usepackage{amssymb}
\usepackage{graphicx}
\usepackage{multicol}
\usepackage{tabularx}
\usepackage{tikz}
\usetikzlibrary{arrows,shapes,automata,petri,positioning,calc}
\usepackage{hyperref}
\usepackage{tikz}
\usetikzlibrary{matrix,calc}
\usepackage[margin=0.5in]{geometry}
% ---- power functions -----% 
\providecommand{\norm}[1]{\left\lVert#1\right\rVert}
\newcommand{\myvec}[1]{\ensuremath{\begin{pmatrix}#1\end{pmatrix}}}
\let\vec\mathbf
%\providecommand $${\norm}[1]{\left\lVert#1\right\rVert}$$
\providecommand{\abs}[1]{\left\vert#1\right\vert}
\let\vec\mathbf

\newcommand{\mydet}[1]{\ensuremath{\begin{vmatrix}#1\end{vmatrix}}}
\providecommand{\brak}[1]{\ensuremath{\left(#1\right)}}
\providecommand{\lbrak}[1]{\ensuremath{\left(#1\right.}}
\providecommand{\rbrak}[1]{\ensuremath{\left.#1\right)}}
\providecommand{\sbrak}[1]{\ensuremath{{}\left[#1\right]}}
%-------end power functions----%
\newenvironment{Figure}
  {\par\medskip\noindent\minipage{\linewidth}}
  {\endminipage\par\medskip}
\begin{document}
%--------------------name & rollno-----------------------
\raggedright \textbf{Name}:\hspace{1mm} Ballepu dheeraj kumar \hspace{3cm} \Large \textbf{Matrices Using Python}\hspace{2.5cm} % 
%\normalsize \textbf{Roll No.} :\hspace{1mm} FWC22067\vspace{1cm}%
\begin{multicols}{2}

%----------------problem statement--------------%
\raggedright \textbf{Problem Statement:}\vspace{2mm}
\raggedright \\Consider a circle with it's centre lying on the focus of parabola $y^2=2px$.such that it touches the directrix of the parabola.The point of intersection of the circle and parabola is ?\\
\vspace{5mm}
%-----------------------------solution---------------------------
\raggedright \textbf{SOLUTION}:\vspace{2mm}\\

%---------given----------------%
\raggedright \textbf{Given}:\vspace{2mm}\\
%Equation of circle is \\\vspace{1mm}
%\begin{align}
%x^2 + y^2 = 16
%\end{align}
Equation of Parabola is \\ \vspace{1mm}
\begin{align}
y^2 = 2px 
\end{align}
From (1) we can say that Parabola is concave towards positive x axis.\\
For the given equation of parabola,
	\begin{equation}
  \vec{V} = \begin{pmatrix}
          0 & 0 \\
          0 & 1 \\
          \end{pmatrix}, \hspace{4mm} \vec{u} = \myvec{-p \\ 0} \hspace{2mm} \& \hspace{2mm} f=0
 \label{eq-2-}
  \end{equation}
 The eigenvalue decomposition of a symmetric matrix $\vec{V}$ is given by
  \begin{equation}
  \vec{P}^{\top}\vec{V}\vec{P} = \vec{D} \hspace{1cm}
  \vec{P} = \myvec{\vec{P_1} & \vec{P_2}}                                                    \label{eq-3-} 
  \end{equation}
\begin{equation}
 \vec{D} = \begin{pmatrix}
          \lambda_1 & 0 \\
          0 & \lambda_2 \\
          \end{pmatrix}
  \label{eq-4-}
  \end{equation}
  On solving $\eqref{eq-3-}$ with $\vec{P_1} = \myvec{1 \\ 0} \hspace{2mm} \& \hspace{2mm} \vec{P_2} = \myvec{0 \\ 1}$, we get
  \begin{equation}
 \vec{D} = \begin{pmatrix}
         0 & 0 \\                                                                              0 & 1 \\                                                                              
 \end{pmatrix}                                                                      
\label{eq-5-}
\end{equation}
where,
  \begin{equation}
  \lambda_1 = 0 \hspace{2mm} and \hspace{2mm} \lambda_2 = 1
  \label{eq-6-}
  \end{equation}
  We have,
  \begin{equation}
  eccentricity, \hspace{3mm} e = \sqrt{1-\frac{\lambda_1}{\lambda_2}}
  \label{eq-7-}
  \end{equation}
  from \eqref{eq-6-},
  \begin{equation}                                                                             e =1
  \label{eq-8-}         
  \end{equation}
  Normal vector of diretrix $\vec{n}$ is given by                                              \begin{equation}                                                                    
\vec{n} = \sqrt{\lambda_2}\vec{P_1}
  \label{eq-9-}
  \end{equation}
  This gives,
  \begin{equation}
  \vec{n} = \myvec{1 \\ 0}
  \label{eq-10-}
  \end{equation}
  For e=1, we have
  \begin{equation}
c=\frac{\norm{\vec{u}}^2 - \lambda_2 f   }{2\vec{u}^{\top}\vec{n}}
  \end{equation}
  On solving we get,
\begin{equation}
 c =  -p/2
\end{equation}
  Focii of a conic is given by the equation,
 \begin{equation}                                                                      
 \vec{F} = \frac{ce^2\vec{n}-\vec{u}}{\lambda_2}                                         
 \label{eq-13-}                                                                          
 \end{equation}
 \begin{equation}                                                                       
 \vec{F} =    \myvec{p/2 \\ 0}                                                        
 \label{eq-14-}                                                                         
 \end{equation}
 The directrix of the parabola is\\
 \begin{equation}
	 \vec{n}^{\top}\vec{X}=\vec c
 \end{equation}
\begin{equation}                                    \begin{pmatrix}                                1 & 0 
\end{pmatrix}\vec{X}=\vec c
\end{equation}
%whose focus is\\
%\begin{align}
%\vec{F}=\myvec{p/2\\
%0}
%\end{align} 
%which is also the centre of the circle
 \vspace{2mm}
%From equation (1) radius of circle is,\\ \vspace{1mm}
%\begin{align}
%r= 4
%\end{align}

%-------------To find ------------------%
\textbf{To Find }\vspace{2mm}\\
To find the intersection points of the circle and parabola\vspace{2mm}  \\ 
%--------------steps----------------------%
\textbf{STEP-1}\vspace{2mm}\\
The given circle and parabola can be expressed as conics with parameters,\\ \vspace{1mm}
For circle,\\ \vspace{1mm}
\begin{align}
\vec{V}_1=\vec{I}
\end{align}
\begin{align}
\vec{V}_1=\myvec{
1 & 0\\
0 & 1
}
\end{align} 

\begin{align}
\vec{u_1}=\myvec{-p/2\\
0}
\end{align} 
\begin{align}
f_1=-3p^2/4
\end{align} \vspace{2mm}

For Parabola,\\\vspace{1mm}
\begin{align}
\vec{V}_2=\myvec{
0 & 0\\
0 & 1
}
\end{align} 

\begin{align}
\vec{u_2}= \myvec{
-p\\
0
}
\end{align} 
\begin{align}
f_2=0
\end{align} \vspace{2mm}

\textbf{STEP-2}\vspace{2mm}\\
The intersection of the given conics is obtained
as\\
\begin{align}
	\vec{x}^{\top}\brak{\vec{V}_1 + \mu\vec{V}_2}\vec{x}+2 \brak{\vec{u}_1+\mu \vec{u}_2}^{\top} \vec{x} 
	\\
	+ \brak{f_1+\mu f_2}= 0
    \end{align}
    
\begin{align}
\vec{V}_1+\mu\vec{V}_2= \myvec{
1 & 0\\
0 & \mu+1
}
\end{align}
\begin{align}
\vec{u}_1+\mu\vec{u}_2= -\myvec{
p/2+\mu p\\
0
}
\end{align}
\begin{align}
f_1+\mu f_2= -3p^2/4
\end{align}
This conic is a single straight line if and only if, \\ \vspace{1mm}
\begin{align}
\mydet{\vec{V}_1 + \mu\vec{V}_2 & \vec{u}_1+\mu \vec{u}_2\\ \brak{\vec{u}_1+\mu \vec{u}_2}^{\top} & f_1 + \mu f_2} &= 0
\end{align}
And,\\
\begin{align}
%\mydet{\vec{V}_1 + \mu\vec{V}_2} &= 0
\end{align}
Substituting equation (25),(26) and (27) in equation (28)\\ \vspace{1mm}
We get,\\ \vspace{1mm}
\begin{align}
\implies \mydet{1& 0 & -p/2-\mu p\\ 
0 & 1+\mu & 0 \\
-p/2-\mu p & 0 & -3p^2/4
} &= 0
\end{align}
Solving the above equation we get,\\ \vspace{1mm}
%\begin{align}
%9\mu^3 + 9\mu^2 + 16\mu + 16=0
%\end{align}
gives,\\ \vspace{1mm}
\begin{align}
    \mu = -1
\end{align}
 Thus, the parameters for a straight line can be expressed as\\ \vspace{1mm}
 \begin{align}
	\vec{V} &= 
\vec{V}_1 + \mu\vec{V}_2
=\myvec{ 1 & 0 \\ 0 & 0},
\\
	\vec{u} &=
\vec{u}_1+\mu \vec{u}_2
	= \myvec{
p/2\\
0
    }
\\
	f&=-3p^2/4,
	\\
%	\implies \vec{D} &= \vec{V}, \vec{P} = \vec{I}
 %   \end{align}
 \implies \vec{D} &= \vec{V}, \vec{P} = \vec{I}
 \end{align} 
 Thus, the desired pair of straight lines are \\
 \begin{align}
       \myvec{\sqrt{\abs{\lambda_1}} & \pm \sqrt{\abs{\lambda_2}}}\vec{P}^{\top}\brak{\vec
    {x}-\vec{c}} &= 0
 \end{align}
\begin{align}
    \implies\myvec{0 & \pm 1}\vec{x}-\vec{c} &= 0
\end{align}
 \begin{align}
      \text{or, }\vec{x} =\vec{c} + \kappa \myvec{\pm 1 \\ 0}
\end{align}
The points of intersection of the line is given by, \\
 \begin{align} L: \quad \vec{x} = \vec{q} + \kappa \vec{m} \quad \kappa \in \mathbb{R}
\end{align}
 with the conic section, \\
 \begin{align}
         \vec{x}^{\top}\vec{V}\vec{x} + 2\vec{u}^{\top} \vec{x} + f = 0
 \end{align}
are given by \\
 \begin{align}
 \vec{x}_i = \vec{q} + \kappa_i \vec{m}
 \end{align}
%with the conic section 
%\begin{align}
%	\vec{x}^{\top}\vec{V}\vec{x} + 2\vec{u}^{\top} \vec{x} + f = 0
%\label{eq:conic_quad_form}
%\end{align}
%\begin{align}
%	\vec{x}^{\top}\myvec{
%1 & 0\\
%0 & 0
%}\vec{x}+2\myvec{p/2 & 0}\vec{x}-3p^2/4=0
%\end{align}
%\begin{align}
 %  q=\myvec{p/2\\-3p/2} 
%\end{align}
%\begin{equation}
%	\vec{x}^{\top}
%	\begin{pmatrix}
%		4 & 0\\
%		0 & 0
%	\end{pmatrix}
%	-2
%	\begin{pmatrix}
%		-2p\\
%		0
%	\end{pmatrix}
%	-3p^2=0
%\end{equation}
%By solving the above equation we get\\
%\begin{align}
%	q=\myvec{p/2\\-3p/2}
%\end{align}
%\noindent Thus, the desired pair of straight lines are \\
%\begin{align} 
%	\myvec{p/2 \\-3p/2 }x+ \myvec{0 &1}y
%	
%\end{align} 
%upon substituting from  The points of intersection of the line 
%\begin{align}
%L: \quad \vec{x} = \vec{q} + \kappa \vec{m} \quad \kappa \in \mathbb{R}
%\label{eq:conic_tangent}
%\end{align}
%\begin{align}
%\vec{x}_i = \vec{q} + \kappa_i \vec{m}
%\label{eq:conic_tangent_pts}
%\end{align}


 %with the conic section  we have ,

%\begin{align}
%\vec{x}_i = \vec{q} + \kappa_i \vec{m}
%\end{align}
where, \\
{\tiny
\begin{multline}
\kappa_i = \frac{1}
{
\vec{m}^T\vec{V}\vec{m}
}
\lbrak{-\vec{m}^T\brak{\vec{V}\vec{q}+\vec{u}}}
\\
\pm
\rbrak{\sqrt{
\sbrak{
\vec{m}^T\brak{\vec{V}\vec{q}+\vec{u}}
}^2
-
\brak
{
\vec{q}^T\vec{V}\vec{q} + 2\vec{u}^T\vec{q} +f
}
\brak{\vec{m}^T\vec{V}\vec{m}}
}
}
\end{multline}
}
On substituting\\
\begin{align}
\vec{q} &= \myvec{
p/2\\
-3p/2
} 
\end{align}
\begin{align}
\vec{m} = \myvec{0 \\ 1}
\end{align}
With the given circle\\ 
\begin{align}
	\vec{V} &= \myvec{
1 & 0\\
0 & 1
    }
\end{align}
\begin{align}
	\vec{u} = \myvec{-p/2 \\0}
 \end{align}
 \begin{align}
  f = -3p^2/4
 \end{align}
The value of  $\kappa_i$\\
\begin{align}
    \kappa_i = 5p/2,p/2
\end{align}
\vspace{2mm}
The points of intersection with Parabola along circle are \\
\begin{align}
    \vec{A}=\myvec{
p/2\\
p
    }
\end{align}
\begin{align}
    \vec{B}=\myvec{
p/2\\
-p
    }
\end{align}

   \includegraphics[scale = 0.4]{/sdcard/Download/Conic/figure1/fig8.pdf}
 

\vspace{2mm} \textbf{Construction}
\begin{center}
\setlength{\arrayrulewidth}{0.5mm}
\setlength{\tabcolsep}{6pt}
\renewcommand{\arraystretch}{1.5}
    \begin{tabular}{|l|c|}
    \hline 
    \textbf{Points} & \textbf{coordinates} \\ \hline
   $\vec{A}$ & $\myvec{
   p/2\\
   p
   } $ \\\hline
   $\vec{B}$ & $\myvec{
   p/2\\
   -p
   } $ \\\hline
      \end{tabular}
  \end{center}

*Verify the above proofs in the following code.\\
\framebox{
\url{https://github.com/ballepu1994}}
\bibliographystyle{ieeetr}
 \end{multicols}
\end{document}
