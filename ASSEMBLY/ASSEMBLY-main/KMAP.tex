\documentclass{article}

\usepackage[english]{babel}

\usepackage[letterpaper,top=2cm,bottom=2cm,left=3cm,right=3cm,marginparwidth=1.75cm]{geometry}

% Useful packages
\usepackage{multicol}
\usepackage{karnaugh-map}
\usepackage{amsmath}
\usepackage{graphicx}
\usepackage{array}
\usepackage{blindtext}
\usepackage[utf8]{inputenc}

\usepackage[colorlinks=true, allcolors=blue]{hyperref}
\title{K map  for Boolean Expression}
\author{Ballepu dheeraj kumar}

\begin{document}
\maketitle
\begin{multicols}{2}
\tableofcontents

\begin{abstract}
 This manual shows how to use Arduino uno and LED to represent the boolean expression.
 
\end{abstract}
\section{Components}

%\begin{table}[]
    \centering
    \begin{tabular}{ |c |c |c |c |}
\hline
\textbf{Components} & \textbf{Value} & \textbf{Quantity} \\
\hline
  
 Arduino & UNO & 1 \\  
 
 
 LED & - & 1 \\
 Jumper wires&M-M &3\\
 Breadboard& &1\\
 \hline
 \end{tabular}
 \vspace{3mm}
 
 %\caption{Table 1.0}
    \label{table1}
%\end{table}

\section{Hardware}

\textbf{Problem 2.1} Make connections between the Arduino and LED using Breadboard.
\section{Software}

\textbf{Problem 3.1.} execute the following program after
downloading.
\framebox{
\url{https://github.com/ballepu1994}}
\vspace{10mm}
%\begin{table}[]

 \vspace{3mm}
 
 %\caption{Table 1.0}
    \label{table1}
%\end{table}
\vspace{5mm}
 \begin{center}
%Truth Table \vspace{2mm} \\
    \setlength{\arrayrulewidth}{0.5mm}\setlength{\tabcolsep}{18pt}
\renewcommand{\arraystretch}{1.5}
    \begin{tabular}{|l|c|r|l|c|}
    \hline % <-- Alignments: 1st column left, 2nd middle and 3rd right, with vertical lines in between
      \textbf{U} & \textbf{V} & \textbf{W} & \textbf{Z} & \textbf{G}\\
      \hline
      0 & 0 & 0 & 0 & 0\\
      0 & 0 & 0 & 1 & 0\\
      0 & 0 & 1 & 0 & 0\\
      0 & 0 & 1 & 1 & 1\\
      0 & 1 & 0 & 0 & 0\\
      0 & 1 & 0 & 1 & 1\\
      0 & 1 & 1 & 0 & 1\\
      0 & 1 & 1 & 1 & 1\\
      1 & 0 & 0 & 0 & 0\\
      1 & 0 & 0 & 1 & 0\\
      1 & 0 & 1 & 0 & 0\\
      1 & 0 & 1 & 1 & 1\\
      1 & 1 & 0 & 0 & 1\\
      1 & 1 & 0 & 1 & 1\\
      1 & 1 & 1 & 0 & 0\\
      1 & 1 & 1 & 1 & 1\\
      \hline
   \end{tabular}\\
   \vspace{3mm}
   Truth Table 
 \end{center} 
 \vspace{30mm}

\begin{karnaugh-map}[4][4][1][$WZ$][$UV$]
        \minterms{3,5,6,7,11,12,13,15}
        \maxterms{0,1,2,4,8,9,10,14}

        \implicant{12}{13}
        \implicant{7}{6}
       \implicant{3}{11}
       \implicant{5}{15}
    \end{karnaugh-map}
%    \vspace{1mm}
    \centering
    \textbf{K-Map}

%\end{flushleft}

U, V, W, Z are the
inputs and LED is the output.Using boolean logic.
\begin{equation}
A= WZ+VZ+UVW'+U'VW
\end{equation}

\end{multicols}{}
\end{document}